%%%%%%%%%%%%%%%%%%%%%%%%%%%%%%%%%%%%%%%%%%%%%%%%%%%%%%%%%%%%%%%%%%%
%--
%-- Documentation of the Hyperlatex 2.6 extensions rhxxname.hlx and
%-- rhxpanel.hlx.
%--
%-- Copyright (C) 2002 Ralf Hemmecke <ralf@hemmecke.de>
%--
%-- $Id: rh-hyperlatex.tex,v 1.1 2004/05/02 20:24:11 tomfool Exp $
%--
%%%%%%%%%%%%%%%%%%%%%%%%%%%%%%%%%%%%%%%%%%%%%%%%%%%%%%%%%%%%%%%%%%%
\xname{hyperlatex/index}
\section{Hyperlatex}\label{sec:Hyperlatex}
\newcommand{\HlxURL}{http://www.cs.uni-magdeburg.de/docs/hyperlatex/}
\newcommand{\HLX}{\Xlink{Hyperlatex}{\HlxURL}}

Recently I rediscovered \HLX{} for me.
%
Since I am fluent in \LaTeX{}, I don't want to bother myself with all
the details in HTML. If I can do my few pages in \LaTeX{} this would
be terrific. So \HLX{} seemed the right tool for me, but not quite.
That is why I offer here two packages for \HLX{} 2.6 and hope someone
finds them useful.

\htmlmenu{1}



%%%%%%%%%%%%%%%%%%%%%%%%%%%%%%%%%%%%%%%%%%%%%%%%%%%%%%%%%%%%%%%%%%%
\xname{hyperlatex/history}
\subsection{History}

My first web-pages I wrote in plain HTML, and they are/were certainly
not very advanced. But now, in order to transfer all my pages into
\HLX{}, I wanted to keep my directory structure (or at least part of
it).
%
Well the first solution one could think of is to replace each of my
HTML files by a corresponding \HLX{} file and add some Makefiles to
generate everything (including subdirectories).
%
Yes, I could have certainly done this way.
%
But then I discovered that \HLX{} offered me (with the \verb'\xname'
macro) a way to choose the target name for the files of each section.
That was fine, so I could put all the files in each of my
subdirectories together into one single file.
%
Additionally, I get the advantage of not having to bother with the
links between the files in one directory anymore because this is now
done in a \LaTeX-like fashion.

So, but why shouldn't I put ALL my web-pages into ONE single \LaTeX{}
file and generate everything just by calling \HLX{}?
%
Wouldn't it be nice to produce all the relative links from one of your
private pages to another using \LaTeX's \verb'\label' mechanism?
%
Further, I know a lot of people who are fluent in \LaTeX{} but are
rather weak in writing Makefiles. Why not offering them an extension
of \HLX{}?

When I started with \HLX{}, I experimented a bit with the
\verb'\xname' command and found out that I could write something like
\verb'\xname{hyperlatex/index}' and \HLX{} would not complain. Well,
nearly, I had to produce the directory \file{hyperlatex} by hand in
order to produce the \file{index.html} file.
%
I was a bit disappointed and asked at the Hyperlatex
\xlink{mailinglist}{mailto:hyperlatex@cs.uni-magdeburg.de}.

One day later Tom Sgouros sent me the following piece of code.
\begin{verbatim}
(defun hyperlatex-format-xname ()
  (if hyperlatex-final-pass
      (hyperlatex-parse-required-argument)
    (let ((node-name (hyperlatex-parse-required-argument)))
      (if (file-name-directory node-name)
          (make-directory (file-name-directory node-name) t))
      (setq hyperlatex-node-names
            (cons (cons (1+ hyperlatex-node-number)
                        (format "%s%s" node-name hyperlatex-html-ext))
                  hyperlatex-node-names)))))
\end{verbatim}
It worked immediately. Directories are then produced automatically.
Thank you Tom.

Eventually, I had to add also a reference to
\verb'hyperlatex-html-directory' so that the directory is generated in
the right place even if someone sets \verb'\htmldirectory' to
something different from the current directory.

Unfortunately, my goal had not yet been achieved completely. The links
that \HLX{} produces are not correct if one generates a whole tree of
subdirectories. The idea how to correct this is simple, but how to
achieve it in Emacs Lisp? My knowledge of Lisp is/was quite limited.

Here I must give my gratitude to Otfried Cheong and all the others who
contributed to \HLX{}. Most things I know now, I have learned from the
code in \file{hyperlatex.el} and \file{siteinit.hlx}.

My contribution (including this documentation) is packaged into
\file{\xxlink{rhxhlx.tgz}{hyperlatex}{rhxhlx.tgz}}. It consists of the
four files \file{rhxxname.hlx}, \file{rhxpanel.hlx},
\file{rhxdoc.tex}, and \file{rh-hyperlatex.tex}.

There are also some PNG buttons in the archive.
%
\textbf{THEY ARE NOT MADE BY ME.}
%
The \file{home.png} icon is a \Xlink{KDE}{http://www.kde.org} button,
but I cannot remember where I found the other pictures. Anyway, I just
provide these icons for demonstration. With a bit of effort everyone
can produce (or find on the Internet) any button he/she likes.










%%%%%%%%%%%%%%%%%%%%%%%%%%%%%%%%%%%%%%%%%%%%%%%%%%%%%%%%%%%%%%%%%%%
\xname{hyperlatex/download}
\subsection{Download and Installation}

Everything is packaged into
\file{\xxlink{rhxhlx.tgz}{hyperlatex}{rhxhlx.tgz}} (including the
documentation).

Put the \file{.hlx} files into a place where \HLX{} finds them. Maybe
your directory \file{{\~{}}/.hyperlatex} is a good place.
%
The buttons should be either placed into a fixed directory
corresponding to a certain URL or into the directory of your web-site.
If \file{/home/www/myname} is your top entry point of your home page
(this corresponds to \verb'\htmldirectory'), put the pictures there.













%%%%%%%%%%%%%%%%%%%%%%%%%%%%%%%%%%%%%%%%%%%%%%%%%%%%%%%%%%%%%%%%%%%
\xname{hyperlatex/copyright}
\subsection{Copyright}

The files \file{rhxxname.hlx}, \file{rhxpanel.hlx},
\file{rhxdoc.tex}, and \file{rh-hyperlatex.tex} are ``free'' in the sense
that everyone is free to use and free to redistribute it under certain
conditions. 

Copyright \copyright{} 2002 \xlink{Ralf Hemmecke}{mailto:ralf@hemmecke.de}

These programs (including documentation) are free software; you can
redistribute it and/or modify it under the terms of the \Xlink{GNU
  General Public License}{http://www.gnu.org/copyleft/gpl.html} as
published by the Free Software Foundation; either version 2 of the
License, or (at your option) any later version.

These programs are distributed in the hope that they will be useful,
but \emph{without any warranty}; without even the implied warranty of
\emph{merchantability} or \emph{fitness for a particular purpose}.
%
See the
%
\Xlink{GNU General Public
  License}{http://www.gnu.org/copyleft/gpl.html}
%
for more details.
\begin{iftex}
  A copy of the GNU General Public License is available on
  the World Wide web at \texttt{http://www.gnu.org/copyleft/gpl.html}.
%
  You can also obtain it by writing to the Free Software Foundation,
  Inc., 675 Mass Ave, Cambridge, MA 02139, USA.
\end{iftex}












%%%%%%%%%%%%%%%%%%%%%%%%%%%%%%%%%%%%%%%%%%%%%%%%%%%%%%%%%%%%%%%%%%%
\xname{hyperlatex/rhxxname}
\subsection{The xname Extension}

The package is loaded after \verb'\usepackage{hyperlatex}' by putting 
\begin{verbatim}
\W\usepackage{rhxxname}
\end{verbatim}
into the preample of your \LaTeX{} file or simply write the above
command (the \verb'\W' is then not necessary) into your
\file{init.hlx} file.

The package extends \HLX{} in such a way that it becomes possible to
generate subdirectories by writing something like
\verb'\xname{DirA/DirB/File}' just before the section that should go
to a separate file. The links from and to all elements in your whole
document will be generated correctly by \HLX{}.

Well, there are a few commands that could perhaps be interesting, but
I don't want to make them generally available to a normal user because
then I would have to rename them or additionally provide an
\file{rhxxname.sty} file.

My hope is that this package becomes redundant in the next version of
\HLX{}.

A great advantage of this extension is that it then becomes possible
to generate all of your web pages from one single \LaTeX{} file
without the need to write Makefiles.











%%%%%%%%%%%%%%%%%%%%%%%%%%%%%%%%%%%%%%%%%%%%%%%%%%%%%%%%%%%%%%%%%%%
\xname{hyperlatex/rhxpanel}
\subsection{Documentation of rhxpanel.hlx}

This package controls the panel generation of \HLX{}.

\htmlmenu{1}

\xname{hyperlatex/load}
\subsubsection{How To Load The Package}
The package is loaded after \verb'\usepackage{hyperlatex}' by putting 
\begin{verbatim}
\W\usepackage{rhxpanel}
\end{verbatim}
into the preample of your \LaTeX{} file.

I would, however suggest to work with
\begin{verbatim}
\usepackage{color}
\W\usepackage{frames}
\W\usepackage{rhxpanel}
\W\usepackage{rhxxname}
\end{verbatim}
in this order. You are, of course, free to put these commands into
your \file{init.hlx} file.

The \file{color} package will be automatically loaded in
\file{rhxpanel.hlx}, but you might want to set colours yourself for
your web-pages.






\xname{hyperlatex/bluepanels}
\subsubsection{bluepanels.hlx}
The package \file{rhxpanel.hlx} can be thought of as a replacement for
the \file{bluepanels.hlx} package. In fact, the latter can be simulated
as follows.
\begin{verbatim}
\definecolor{panelcolor}{rgb}{0.7,0.8,1}
\definecolor{paneltextcolor}{gray}{0}
\newcommand{\HlxIcons}{http://www.cs.uu.nl/{\~{}}otfried/img/}
\newcommand{\HlxIconExtension}{gif}
\newcommand{\HlxPanelHome}{0}
\newcommand{\HlxPanelShade}{0}
\newcommand{\HlxPanelPreviousBeforeNext}{0}
\end{verbatim}








\xname{hyperlatex/color}
\subsubsection{Colour Control}
The colour of the panel can be customised by defining the colours
\emph{panelcolor} and \emph{paneltextcolor} like this which is the
default.
\begin{verbatim}
\definecolor{panelcolor}{gray}{0.7}
\definecolor{paneltextcolor}{gray}{0}
\end{verbatim}
See the color package for more details.













\xname{hyperlatex/buttons}
\subsubsection{The Buttons}
Although this panel is intended to show 4 buttons at the top and
the bottom of each page (the default behaviour), it can be
customised to other styles, see below.

\begin{ifset}{rhxpanelloaded}
The buttons are as follows.
\begin{enumerate}
\item The HOME button \htmlicon{home.\HlxIconExtension}{HOME}.
%
  Go immediately back to the initial page.
%
\item The PREV button \htmlicon{previous.\HlxIconExtension}{PREV}.
%
  Go one page back.
%
\item The UP button \htmlicon{up.\HlxIconExtension}{UP}.
%
  Go up one hierarchy.
%
\item The NEXT button \htmlicon{next.\HlxIconExtension}{NEXT}.
%
  Go one page forward.
\end{enumerate}
\end{ifset}
The behaviour of the PREV and NEXT buttons can be controlled with the
standard \file{sequential.hlx} package of Hyperlatex.

Panel fields can be added like in bluepanels.hlx using the command
\verb'\htmlpanelfield{TEXT}{LABEL}'.
%
To add icons instead of TEXT, use a command similar to what follows.
\begin{verbatim}  
  \htmlpanelfield{\HlxPanelIcon{icon.png}{ALTTEXT}}{LABEL}
\end{verbatim}
This then shows the icon \file{icon.png} which should be in the same
place as the files for the other navigation buttons.
%
Note that \verb'\HlxPanelIcon' is an \verb'\Hlx...' style command and
should be hidden from \LaTeX{}.













The place of the panel buttons is controlled by defining the
command
\begin{verbatim}
  \newcommand{\HlxIcons}{http://www.cs.uu.nl/{\~{}}otfried/img/}
\end{verbatim}
The following setting is the default.
\begin{verbatim}
  \newcommand{\HlxIcons}{}
\end{verbatim}
Whereas the first specifies a fixed place in the Internet, the second
form describes a local directory starting at \verb'\htmldirectory'. A
slash will be added automatically to the path if it is not provided.

The default extension of the button icons is \file{.png}, but you can
change this by defining
\begin{verbatim}
 \newcommand{\HlxIconExtension}{gif}
\end{verbatim}

There are several commands to control the generation of buttons.
Here is a list of commands that you can put into your LaTeX file.
\begin{itemize}
\item \verb'\newcommand{\HlxPanelHome}{0}'\\
  Don't generate the HOME button.
%  
\item \verb'\newcommand{\HlxPanelHome}{1}' (default)\\
  Generate a HOME button.
%  
\item \verb'\newcommand{\HlxPanelShade}{0}'\\
  Inactive buttons are not shown.
%  
\item \verb'\newcommand{\HlxPanelShade}{1}'  (default)\\
  Inactive buttons appear shaded. (The buttons noprevious.png,
  noup.png, nonext.png are used.)
%  
\item \verb'\newcommand{\HlxPanelPreviousBeforeNext}{0}'\\
  The order of the navigation buttons is NEXT, UP, PREVIOUS.
%  
\item \verb'\newcommand{\HlxPanelPreviousBeforeNext}{1}' (default)\\
  The order of the navigation buttons is PREVIOUS, UP, NEXT.
\end{itemize}

The icon images are not part of the package. They are just packaged as
examples. One should provide 7 buttons, home.png, blank.png,
previous.png, up.png, next.png, noprevious.png, noup.png, nonext.png.
The filename extension depends on what you define in
\verb'\HlxIconExtension'.









\xname{hyperlatex/navigationpanel}
\subsubsection{Navigation Panel}
This package also contains some code to control the appearance of
the navigation panel. If you load the frames package before
rhxpanel.hlx then you will have a slightly different appearance
than in the case that you first load rhxpanel.hlx and then
frames.hxl. I suggest the following:
\begin{verbatim}
  \usepackage{color}
  \W\usepackage{frames}
  \W\usepackage{rhxpanel}
  \W\usepackage{rhxxname}
\end{verbatim}
Try it out what you like more or even define your own navigation
panel.












\xname{hyperlatex/addons}
\subsubsection{Additional Commands}

There are two commands that might be useful for navigation.
\begin{itemize}
\item \verb'\HlxTopNodeUrl'\\
  This points to the topnode of the (hyper-)\LaTeX{} file you are
  translating. It corresponds to the top entry point of your file.
\item \verb'\HlxHomeUrl'\\
  By default this is the same as \verb'\HlxTopNodeUrl'. So this
  corresponds to your HOME. If you decide to have separate files and
  you want to let the HOME be something else and still have the home
  button of the panel correctly pointing to your HOME, you should
  redefine this command as follows.\\
\begin{verbatim}
  \newcommand{\HlxHomeUrl}{http://www.my.home.org/my/dir}
\end{verbatim}
\item \verb'\HlxNavigationPanelTitle'\\
  By default this is empty, but you can set this to anything you like
  by saying something like this.
\begin{verbatim}
  \newcommand{\HlxNavigationPanelTitle}{MY TITLE}
\end{verbatim}
\end{itemize}

